\documentclass{article}
\usepackage{graphicx}
\textheight=8.5in
\textwidth=6.5in
\oddsidemargin=0.0in
\topmargin=0.0in


\renewcommand\baselinestretch{1.0}           % single space
\pagestyle{empty}                            % no headers and page numbers
\oddsidemargin -10 true pt      % Left margin on odd-numbered pages.
\evensidemargin 10 true pt      % Left margin on even-numbered pages.
\marginparwidth 0.75 true in    % Width of marginal notes.
\oddsidemargin  0 true in       % Note that \oddsidemargin=\evensidemargin
\evensidemargin 0 true in
\topmargin -0.75 true in        % Nominal distance from top of page to top of\textheight 9.5 true in
       % Height of text (including footnotes and figures)
\textwidth 6.375 true in        % Width of text line.
\parindent=0pt                  % Do not indent paragraphs
\parskip=0.15 true in
\usepackage{color}              % Need the color package









\begin{document}
\begin{center}
\section*{\underline{Documentation:Quiz}}
                         Author Name : Satyapal 
\end{center}
This file provides Documentation for Quiz.
\begin{enumerate}
\item Generate a Quiz. 
\item View a Quiz.
\item Insert questions in Quiz.
\item Delete the Quiz.
\item Check the Answers.
\item Awarded grade/calculate the marks.
\item Record of grades of students.
\end{enumerate}
\subsection*{\underline{Generate a Quiz}}
\begin{enumerate}
\item  Request to generate a quiz.
\item  ACL check permission user-id, module-id, course-id.
\item  If success generate a quiz.
\item  If fails error message.
\item  Sending request to Q.B. for generate quiz.
\item  Returning request.
\item  ACL check.
\item  Returning quiz.
\begin{center}
\input{Generate_Quiz.fig.latex}
\end{center}
\end{enumerate}
\subsection*{\underline{View a Quiz}}
\begin{enumerate}
\item Selecting a quiz for view from the list.
\item Check permission user-id, ACL, course-id.
\item If success.
\item If fails return error message.
\item Request to Q.B for view quiz (question-id).
\item Returning question along with Answers.
\item Store the question \& Answers.Temporaries.
\item Return question.
\end{enumerate}

\subsection*{\underline{Insert question in Quiz}}
\begin{enumerate}
\item Request to insert a question.
\item ACL check permission user-id, course Name for read / write permission.
\item If success.
\item If fails error message.
\item Type of question multiple/ True/ Fill in the blank.
\item ACL check.
\item Type of question to insert.
\item Check (multiple/True/Fill).
\item ACL check.
\item Insert question.
\item Save in Q.B.
\item Submission message.
\input{InsQuest_Quiz.latex}
\label{figure:InsQuest_Quiz.latex}
\end{enumerate}
\subsection*{\underline{Delete quiz}}
item Interact with web app.
Sending request to delete a quiz.
ACL check (user-id).
If success delete.
If Fails error message.
Send message.
\begin{center}
\input{Delete_Quiz.latex}
\label{figure:Delete_Quiz.latex}
\end{center}

\subsection*{\underline{Check the answers}}
\begin{enumerate}
\item Submit the Answer for checking.
\item ACL check (user-id) (course name) (course-id) (question-id).
\item Request to check Answer with original Answer. Checking (question-id).
\item Matching Answers.
\item Store the scores in database in quiz.
\item Return the score \& wrong Answer.
\item ACL check.
\item Return the score.

\begin{center}
\input{CKAnswer_Quiz.latex}
\label{figure:CKAnswer_Quiz.latex}
\end{center}
\end{enumerate}

\subsection*{\underline{Awarded grade / calculate the marks}}
\begin{enumerate}
\item User interacts with web app.
\item Request for getting grade/marks.
\item ACL check (question-id), (user-id), (course name).
\item Sending request (If success).
\item Return request.
\item ACL check.
\item Returning the Awarded grades.
\begin{center}
\input{Awarded_Quiz.latex}
\label{figure:Awarded_Quiz.latex}
\end{center}
\end{enumerate}
\subsection*{\underline{Records of grades of students}}
\begin{enumerate}
\item Interact with web app.
\item Requested for Records of grade.
\item ACL check (user-id), (question-id), (course name).
\item Sending request(if success).
\item Returning the Records.
\item ACL check (question-id).
\item Records of grade.
\end{enumerate}

\subsection*{\underline{API(quiz)}}
\begin{enumerate}
\item Generate a Quiz \\
\item Request-generate Quiz User-id Course-id Module-id.
\end{enumerate}
\begin{itemize}
\item Public generate quiz.
      /* This class is used to generate a quiz based on course-id \& module-id.
\item Strings check permission.
      /* Check course-id valid or not.
      /* user-id
\item String course-id {/*get all quiz of particular id from question bank}
      /* get course name.
\item String fail error message.
\item Return message.
\item Return quiz.
\end{itemize}

\subsection*{\underline{View a Quiz}}
\begin{enumerate}
\item View quiz (course-id, question)
\begin{itemize}
\item public view quiz


/* This is used to view a particular quiz.

\item String course-id.

/* It will get particular course-id from question bank.

\item String question-id.

/* It will show the question from question bank.

\item String return question.

/* It will return question. 

\item String return Answer.
\* It will return true answer from question bank for temporary Storage in quiz for checking.
return message.  
\end{itemize}
\end{enumerate}

\subsection*{\underline{Insert a question in quiz}}
\begin{enumerate}
\item Insert question(user-id, course name, course-id)
\begin{itemize}
\item Type of question(M/F/T) 
\end{itemize}    
\begin{itemize}
\item Public Insert question. 

\item/* It will insert question.

\item String type of question. 

\item/* To get the type question (M/T/F).

\item String check permission.

\item/* To check permission.

\item String save question.

\item /* Save question into question bank.

\item String Subsection message.

* return message.

\end{itemize}
\end{enumerate}

\subsection*{\underline{Delete Quiz}}
\begin{enumerate}
\item Delete quiz (user-id, course-id)
//get message

\item Public Delete quiz.

\* This class is used to delete a quiz.

\item String check permission.

\* to check permission.

\item Return message().
 
\*if Success or fail.

\end{enumerate}

\subsection*{\underline{Check Answers}}
\begin{enumerate}

\item Check answer(Course-id, Question-id).
\begin{enumerate}
\item Public check answer.
\* it is used to check answer.
\item String get answer().
\* It will call the true answer from quiz for check.
\item String match Answer().
\* it will match the Answer with true answer sheet.
\item Return Score().
\* this will return score return wrong answer.
\* it return wrong answer.
\item Get true answer().
\* it will get true correction of wrong answer.
return message.
\end{enumerate}
\end{enumerate}

\subsection*{\underline{Awarded grades/calculate the marks}}
\begin{enumerate}

\item Grades (course-id, user-id).
\begin{enumerate}
\item Public grades.
\* This class is used to award grades to student for quiz.
\item String check permission().
\* to check permission.
\item Return grades()
\* after calculation marks.
\end{enumerate}
\end{enumerate}

\subsection*{\underline{Records of grades of student}}
\begin{enumerate}
\item Records of grades(course-id, user-id).
\item Public recordes/grades.
\* This class stores the records of grades of student.
\item String check permission.
\* check permission(user-id, course-id).
\item String return records.
\end{enumerate} 
Written By : Rekha Pal
\end{document}
