\documentclass{article}
\usepackage{graphicx}
\textheight=8.5in
\textwidth=6.5in
\oddsidemargin=0.0in
\topmargin=0.0in

\renewcommand\baselinestretch{1.0}           % single space
\pagestyle{empty}                            % no headers and page numbers
\oddsidemargin -10 true pt      % Left margin on odd-numbered pages.
\evensidemargin 10 true pt      % Left margin on even-numbered pages.
\marginparwidth 0.75 true in    % Width of marginal notes.
\oddsidemargin  0 true in       % Note that \oddsidemargin=\evensidemargin
\evensidemargin 0 true in
\topmargin -0.75 true in        % Nominal distance from top of page to top of\textheight 9.5 true in         % Height of text (including footnotes and figures)
\textwidth 6.375 true in        % Width of text line.
\parindent=0pt                  % Do not indent paragraphs
\parskip=0.15 true in
\usepackage{color}              % Need the color package
\usepackage{epsfig}

\begin{document}
\begin{center}
\section*{\underline{Synchronous Virtual Classroom Tool}} 
 
                                     Specifications 
 
                                      Ver.0.1.1 
 
                                        Authors: 
 
                               Dr.Yatindra Nath Singh 
                        Educational Technology Research Group 
                                       IIT Kanpur 
 
                                         and 
 
                                       Aptech 
 
                                 13 August 04 1245hrs 
 
No part of this document can be reproduced in any form without the prior written permission from the authors of this document. Sharing of this document with any third party which will lead to monitory benefits of any kind is not permitted unless both ETRG, IIT Kanpur and Aptech agrees for any such sharing of document with the third party.
\end{center}
\subsection*{\underline{Introduction}}
\begin{enumerate}
\item[{}{}]
  
Synchronous learning through a live virtual classroom environment enables institutions/ universities to deliver live learning to students via the Internet or 
intranet.

 Traditional Instructor-led classrooms have been fundamental to effective learning for thousands of years. Virtual classroom environment and tools are created in an effort to simulate the interactive nature of a real classroom in a multi campus environment. 

 The virtual classroom environment discussed here will be used to deliver a live lecture. In a Live lecture, teacher delivers a lecture which is transmitted to all the attendees on multicast network live  i.e., in real-time.
 
The virtual classroom will function in a bi-directional mode, i.e. teacher will deliver the lecture and the student can may also interact with the teacher by posing queries and responding to queries posed by the teacher.
\end{enumerate}

\subsection*{\underline{Session Announcement}}
\begin{enumerate}
\item[{}{}]
This should be part of LMS i.e.,web application as I understand(some parts marked Yes)
\begin{enumerate}

\item The teacher will request for a virtual classroom (VC) session. The teacher sends a notification to the system administrator for using the Virtual Classroom with the following details: (yes) 

\item Name of the lecture topic, Course id+Student Batch id in which the lecture will be conducted, (OK, Batch id is to be included as an extension to the course id, as typically a batch of students would be attending a VC session simultaneously)

 \item Session Objectives, 

\item Schedule and time for the session, 

\item Duration of the session, 

\item Name of the Speaker, 

\item Whether FAQs need to be used for this session for answering student queries, which exist in the database, 

\item List of student is not required as suggested in the orginal document, as all the people who has registered as students in the the course id will be able to attend the session. (Yes) 

\item The administrator checks the resource availability and accepts / rejects the request. (this is the resource management part of  the VC) 

\item The administrator sends appropriate email for acceptance/ rejection to the teacher. (this is the resource management part of  the VC) 

\item The administrator creates a session entry. (He enables the participants with the course id and at the correct time to access the course) for the session;  
The administrator sends a confirmation email to the teacher. 

\item On receipt of confirmation, an automated mail is sent from the teacher's end to all the participants for the event. (This can be automatically notified by the LMS, so this item is not required.) Yes, LMS does it. 
\end{enumerate}
\end{enumerate}
\subsection*{\underline{Session Entry}}
\begin{enumerate} 
\item[{}{}]

At the designated event time /date, all the participants should be allowed to authenticate themselves and enter the Virtual Classroom through their client systems. Based on their role, each user will be able to see the appropriate interface/tools. (E.g. Student/ Teacher)  

(Admin role is not required here due to multicast nature of the transmission). ? Yes  
\end{enumerate}
\subsection*{\underline{Features of the Teacher's Interface}}
\begin{enumerate}
\item[{}{}]
The teacher should be able to: 
\begin{enumerate}

\item Use the whiteboard (piece of software on PC) to write text and draw as naturally as in the real classroom.
\item  This requires use of electronic pens interfaced with the PC. (tablet PC, electronic board, web camera are alternative systems ? web cam may be the least expensive system. What about resolution? Teacher systems are limited and we could use any system. Students may use web cam or inexpensive tablet)
 
\item Invite a participant to use the whiteboard. 
\item Project an image on the whiteboard. Animation/ movie/ ppt/ pdf etc. to be viewed by all the participants can be ftped in multicast mode to all the participants. And these can be played by the students through appropriate software when instructed by the instructor. In phase-2, we can enable remote process initiation of appropriate software to play the ftped files.
 
\item Put a question to a participant or to the entire class. When the query is put to the entire class, students will raise hand and teacher can choose the students to answer the question. 
\item View the list of all participants in a virtual class. 
\item See when the student entered/left the class. 
\item View all queries(voice/text) put up by the participants, sorted on a priority. 
\item Answer student queries. The teacher may be able to choose the mode of answering query. 
\item Anytime during the lecture (no. of queries based on a pre-set limit). 
\item At the end of the lecture. 
\item Set a limit on the number of queries he/she wishes to take at a time during the lecture. 
\end{enumerate}  
\end{enumerate}
\subsection*{\underline{Query Handling Process}}
\begin{enumerate}
\item[{}{}]

The queries posed by the student/teacher may be either in text or voice with limited graphics. Similarly the responses given by the teacher/student may also be in text or voice with limited graphics.

The Control logic will function in the manner described below:
 
Any text query posted by the user will be answered by following means : 


\begin{enumerate}

       \item Answered directly through the FAQs database without teacher intervention. 

        \item Dropped, if found irrelevant 
        
         \item Answered by the teacher during/after the lecture at appropriate time/stage. 

         \item Answered posted to a newsgroup/sent via email to all students. 

Any text query posed by the student will be directed first to the FAQs database by the control logic. For this an FAQs database of relevant topic must be available and enabled. 
 
The FAQs database will have a set of metatags (most probably objectives \& type of objectives) so that they can be categorized for selection of questions based on the lecture topic. (Relevant FAQs are derived from the objectives for the lesson) 
 
When the session is created/announced by the teacher, an option to use the FAQs database is provided to the teacher. If the teacher decides to use the FAQs database, a certain section of questions, which are related to the lecture, are categorized and indexed for use by the system. (This will be done via LMS ?  What we create is a partial image of the FAQ for the lesson. I feel it is better if it is within VC for obvious reasons ? reduce the number of transactions.) 
 
If the response to the query exists in the database it will be presented to the student directly. The student while getting the response will be indicated that his query was answered through the FAQs database. 
 
If the response does not exist in the database then further decisions will be taken by the control logic on the nature of the question 

All the text questions will be put in text-query queue and all the voice question will directly queued in voice query queue. 
 
In the text query queue the queries will  be ordered with higher priority for relevant and important queries. The trivial queries will have low priority. If the query is totally irrelevant to the discussion topic, the query will be dropped by the system and an appropriate message/indication will be sent to the student who posed the query by the system. 
 
(These question if not answered by the instructor can be posted in newgroup, where other student might answer them or instructor might give the answer if feels necessary or he can intervene if the discussions among student goes in incorrect direction.) (Jyoti ? Consider also ** below - unanswered questions are not issues that need discussion. Further these questions would not be complete in any sense \& they are context dependent. Hence they are answered by the teacher and the mail with responses goes to all the students. Newsgroup responses are generally context independent.)   

The instructor depending on his choice can pick the question from voice queue or text queue. He may also choose not give answer to any queries.  
 
All the unanswered queries will be posted to newsgroup of the course and will become open to discussion among students. ** 
 
\# 
In the queue, the participant who put the query has an option to drop the question if he feels his query is already answered and is not required to be answered again by the teacher.  
 
Optionally, the teacher may choose to take queries within the lecture by provisioning for handraising capability. The handraising capability can be switched off or on by instructor. 
\begin{center}
\input{brihSync_Query_hendling_pro.latex}
\label{figure:brihSync_Query_hendling_pro.latex}
\end{center}
\end{enumerate}
\subsection*{\underline{Student Interface}}
\begin{enumerate}
\item Hardware Interfaces
\begin{itemize}
\item A Multimedia Computer with speaker and microphone.  
\item A 64kbps ISDN/ xDSL/ Cable connection to connect to the host server. 
\item A light pen or a digital tablet type of device to be used on the interactive board. 
\item Web cam.
\end{itemize}
\item
Application Interface
\begin{itemize}
\item The student's viewing area will be divided into 3 frames mainly all of which are customizable 
\item Video frame showing the teacher (could be cut down if bandwidth not enough, later on can be replaced by layered codec.) 
\item The interactive whiteboard. 
\item An input frame from where the student can post a text/ voice query or reply to a question asked by the teacher. 
\item Control and status window to control audio, video and to display various parameter related to multicast connections. Media streams may be automatically added or deleted depending on bandwidth available. 

 In addition, the student's view will also have the following facilities. 

\item View the participant's list. 
\item Color Status Indicators for each query to indicate: queue in which it is waiting. After the query is answered, the indicator for the relevant query disappears. This is only for instructor. Students need not see what questions have been posted.
\item Editing tools for using the interactive board, on invitation by teacher. (Draw line, circle, ellipse, rectangle, zoom, Change Color, Type Text, Change Font Size, Font color, Erase Tool). In future, if need is felt options for Cut, Copy, Paste and additional functions can be provided as a part of this tool. 
\item replay a session after downloading the recording from LMS.
\item Login/Logout to/from a session.
\end{itemize} 
\end{enumerate}
\subsection*{\underline{Teacher's Interface}}
\begin{enumerate}
\item 
Hardware Interface
\begin{itemize}
\item A Multimedia Computer with wireless headphones and microphones for the teacher's use. 
\item A pen mouse or Tablet PC. 
\end{itemize}
\end{enumerate}
\subsection*{\underline{Application Interface}}
\begin{enumerate}
\item Login/Logout to/from a session. 

\item View of the participant's list. Each participant's name will be a clickable entry,which can be further clicked to view the profile of the participant. Against the participant's list there will be a column where the student's answers will be displayed when the teacher poses a question to participants. (Displaying the student's list and the responses in front may not be  needed. Only one student responds at a time and so it is adequate if the responder's name and the response are displayed in a separate window.For opinion poll kind of questions, a category wise(pro/anti kind of) sum total of the student responses may be displayed to the teacher. Individual student responses need not be shown.)
   
\item An input frame from where the teacher can post a text/ voice query to be answered by the participants. 

\item Select mode for handraising control exerted. (interrupted/uninterrupted). 

\item Query Queue (Query String, Name of the person who posted the query).

\item Controls to switch control of the interactive board to a particular student. 

\item Project an image, animation, video, ppt, pdf etc to be viewed by all participants. 

\item Upload a file for download by participants. 

\item Record and archive a session. This will be done automatically and will stored in LMS using ftp server in LMS area.  

 Control window to modify the audio, video parameters and signaling. (Details will be worked out during design process).
 
 
IF\#1, IF\#2 to be mutually decided by the ETRG, IITKanpur and Aptech teams, to be implmented in LMS part.(OK) 
 
FTP server to be implemented/incorporated by LMS development team. (OK) 
 
Text query classifier to be implemented by Aptech, interface to be provided which will be used to by application disable or enable the classifier depending on whether implementation exists or not.  
 
Database schema to be provided in xml format compliant to Torque framework. (will be worked out jointly). 
\end{enumerate}

\subsection*{\underline{Identified functionalities of the system}}    
\begin{enumerate}      
\item  Enabling/Disabling of interrupt flag (Instructor). 
\item HandRaising (Students). 
\item Voice querry withdrawal (Students). 
\item Granting permission for interaction (Instuctor). 
\item Voice query submission (Students). 
\item Withdrawing permission (Students). 
\item Withdrawing Raised Hand (Students). 
\item Instructor can force to drop the raised hand (Instructor). 
\item Blocking specific users from raising hand and queries (Instructor). 
\item Joining a lecture session (Student). 
\item Authentication (Student and Instructor). 
\item Text query preparation (Students). 
\item Text query submission (Students). 
\item Transfering files to all students (Instructor). 
\item Getting already sent files from server (Student). 
\item Bandwidth estimator (instructor \& students). 
\item  Multicast stream join \& prune signalling (Bandwidth estimator). 
\item  Starting lecture session (Instructor). 
\item Logout from  the system (Student and Instructor).  
 
Note:- Corba or RMI to be used for client server communication for security. 

Detailed scenarios with components is being worked out and will be updated in specification documents as design progresses. Modification in the basic design will be done as per requirements and will be intimated.

\begin{enumerate}
\item[{}{}]
Method-1: 
 
void getLogin(Login,Password) (Object name = GUI) 

/*get login and password of the user*/ 
 
String  encryptPassword(Password) (Object name = utils) 

/* Encrypt the Password*/ 
 
Method-2: 
 
String  sessionKey=verifyLogin(Login,Password,host) (Object name = server) 

/*send the login \&password to the server skelton.  At the server we check the authentication. If it fails then call 3(i) else 4*/  
 
Method-3: 
 
(i) If (sessionKey==null) call 3(ii) else (4) 

void displayMessage(text1)
   
 /*Take the value of text1 from properties file */    (Object name= GUI) 

void displayMessage(text2) 

 /*Take the value of text2 from properties file  */    (Object name = GUI) 
 
Method-4:
 
String addAutherizedUser(Login,StringIP) (Object name = sessionmanager) 

/*Save Information in a XML file or database or in text file*/ 

Method-5: 

void setLoggerForAuthUser(String Login,StringIP) (Object= logger) 

/*Info. such as login=?,IP=?,Key=? etc is written in the file with current date and the login time*/   

Method-6:
 
void setUserInformation( sessionKey ) (Object=client) 

/*here set the userId ,sessionKey of  the user  */
\end{enumerate}
\end{enumerate}
 
\subsection*{\underline{After Authentication getting CourseString,UserInformation  And SessionListing}} 
\subsection*{\underline{with all the Information in which student is registered.}}   
\begin{enumerate}
\item[{}{}]


Method-1: 
 
 Vector courseList = getCourseListing(String key, String IP, int UserId) (Object= server) 
	/*Send the request to the server to get the course list*/ 

	/*Here we first check this user is authorized or not.  go to 1(a)  else 1(c) 
 
(a)   call utils                                                                                                       (Object= utils) 
boolean ComparesKey(File f, String clientKey, String IP)    

	/*Here we check the authorization. If true go to (b) else (c) 

(b)   courselist=Courselist(int userid)                                                           (Object= server) 

      /*get course list with his/her role such as he/she is student or instructor*/ 
 
(c)      return null 
 
Method:-2 
 
String CourseName = getCourse(courselist) 

   /*get the coursename from the client */   (Object=GUI) 
 
Method:-3 
 
 Vector SessionListing =getSessionListing(String Key, String IP, String Coursename)(Object=server) 
 
(a)  call utils                                          (Object=Utils) 

boolean comparesKey(File f, String clientKey, String IP) 

/*If true go to (b) else (c)		 

(b) SessionListing=getList(String Coursename)   (Object=Server)
               /* get session list corresponding to the course name 
 
(c) return false  
 
Method:- 4 

void showSessiondescription (string sessionname)    (Object=Gui)

/* using Jtool tip or Jtool Tip Manager, we show the session DescriptionCorresponding to that sessionname or lecturename when client's mouse over the sessionname */

\end{enumerate}
\subsection*{\underline{Joining a Session}}
\begin{enumerate}
\item[{}{}]

/* When Client press the Run Button */

\begin{enumerate}	 

Method:-1
 
String sessionName = getSessionName()  (Object=GUI)
 
     /*Select SessionName from the  client*/ 

Method:-2 

String clock=clockSynchronize(sessionname) 

     /*Get the server Date and Time */  (Object=Server) 

     (a) String DateTime=dateTimeofServer()         (Object=Server) 

     (b) String  DateTimeofSession=dateTimeof Session(sessionname)     (Object=Server) 
     (c) boolean   result=comparissonofDateTime(DateTime,DateTimeofSession)     (Object=Server) 
        /*  If result is false goto(d) else (f)  */ 

     (d) String timeOfDifference=TimeofDifference(DateTime,DateTimeofSession)    (Object=utils) 

     (e) return timeDifference 

     (f)  return Result 

Method:-3 

/* If(clock is true )  go to (4)   */ 

       else String displayMessage(text3)             (Object=Gui)     

Method:-4 

     int sessionId=getSessionId(String sessionName)   (Object=Client) 

Method:- 5 

 String key =generate multicastkey (int session Id)             (Object=Server) 

(a)  /* Check the MulticastKey is assign or not to this session Id.If not then goto (b) else(e) */ 

Boolean check sessionkey (sessionId)     (Object=Server) 
 
(b)  /* get the highest SessionId /* 
       int SessionId=gethighest key( )                          (Object=Server)						 
   (c)  /* get the Multicast key of higest SessionId  /*	 
  string key=getMulticast key(SessionId)                             (Object=Server) 
 
(d) String newMulticast key=generateMulticast key(key)                              (Object=Server) 
(e)     return newMulticast key 
\begin{center}
\input{brihSync_Joining_Session.latex}
\label{figure:brihSync_Joining_Session.latex}
\end{center}
\end{enumerate}
\subsection*{\underline{After Joining a Session}}
\begin{enumerate}
\item[{}{}]
Open a new Desktop Application inWhich there is a wb/handraise/chat/video/Audio 

Method:- 1 

1- void StoreLoggerInfoRoom(String username, String Multicast Key,                      
String SessionKey, String host)             (Object=Logger)	 
 
 (a) call utils 

               boolean ComparesKey(file F. String SessionKey, String host)                                                                             
                                                        (Object=Utils) 
 /* If success go to (b)  */ 

(b) void writeLoggerInfo(String username, int usernamestatus, File 

/* Her Filename is SessionName+CurrentDate.txt  username=?   Handraisestatus= Blocked=?      Sessionstatus=?   QueryAllow=?  */              (Object=Logger) 
 
 (c)    void generateMulticastKey(No. of Tools, name of Tools, Key) 
  
           /*  generate no of keys from one key \&  assign to different tools  */      (Object=Utils) 
\begin{center}
\input{brihSync_Afterjoining.latex}
\label{figure:brihSync_Afterjoining.latex}
\end{center}
\end{enumerate}

\subsection*{\underline{After Leaving a Session Room}}
Method:-2 

(a) void RemoveLoggerInfoRoom(String username, int usernameStatus, File F) 

/*  Her Filename is SessionName=Current date.txt  */     (Object=Logger) 

               update the File \& set  
                               status=0/1                 */ 
 
(b) void updateLoggerInfo(String username,int status,File f )                       (Object=Logger) 

\subsection*{\underline{Withdraw  and Raised Hand}}
\subsubsection*{\underline{Raised Hand}}

\item[{}{}]

Method:-1 

(a)void getraisedhand( )                            (Object=GUI) 

Method:-2 

(a)void signalToServerForAction(string username,string sessionname,int status)        (Object=Server) 

(b)  /* opens a file sessionname + currentDate.txt  */ 
void updateFile (File F, String username,int Raised, int Blocked,int withdraw,int Querystatus)                                              (Object=utils) 
\begin{center}
\input{brihSync_Withdrow_Raisedhand.latex}
\label{figure:brihSync_Withdrow_Raisedhand.latex}
\end{center}

\subsubsection*{\underline{Withdraw Hand}} 
\item[{}{}]

Method:- 3      

(a)void withdrawhand( )                     (Object=GUI)            

Methods:- 4 

(a)void signalToServerForAction(string username,string sessionname,int status)        (Object=Server) 

(b)  /* opens a file sessionname + currentDate.txt  */ 
void updateFile (File F, String username,int Raised, int Blocked,int withdraw,int Querystatus)                 
                                                (Object=utils)
\end{enumerate}
\subsubsection*{\underline{Withdraw or Blocked specific student by Instructor}}
\item[{}{}]

Method:-5 

(a) void getactionforstudent (string username,int action)          (Object=Gui) 

Methods:- 6 

(a)void signalToServerForAction(string username,string sessionname,int status)        (Object=Server) 

(b)  /* opens a file sessionname + currentDate.txt  */ 
void updateFile (File F, String username,int Raised, int Blocked,int withdraw,int Querystatus)                 
                           (Object=utils)

\end{enumerate}
\subsubsection*{\underline{Query Allow,Query not Allow,withdraw or Blocked all student by Instructor }}
 
Method:-7 

void getActionForStudent(action)          (Object=GUI) 

Methods:- 8 

(a)void signalToServerForAction(string username,string sessionname,int status)        (Object=Server) 

(b)  /* opens a file sessionname + currentDate.txt  */ 
void updateFile (File F, String username,int Raised, int Blocked,int withdraw,int Querystatus)                 
                       (Object=utils) 
\subsection*{\underline{Get or Show List,Status of User who already joins a specific session}} 
\begin{center}
\input{brihSync_getor_showlist.latex}
\end{center}

Method:-1 

vector loggerlist=getLoggerStatus(sessionname)                             (Object=Server) 

/* get the Information from the file \&  and show in client GUI using Thread */      \\                 

Method:-2 
                                   void showLoggerList(loggerlist)                                              (Object=GUI)
\subsection*{\underline{Query Preparation \& Submission}}

Method:-1  

 void gettextQuery( )                           (Object=GUI) 
 
/* Get text Query \& QueryId is username + sessionname */ 

Method:-2 

  void getvoiceQuery(string query Id)  (Object=GUI) 

/* Get voice Query which is generated with the help of recorder. 

This  voice Query is a stored in a file which extension is JMF supported  
file */ 

Method:-3 

 void savetextQuery(string query Id)   (Object=GUI)  

  Method:-4                                             

 void savevoiceQuery(string query Id)    (Object=GUI) 

Method:-5 

void retrivetextQuery(String Query ,int QueryId )   (Object=Client) 

Method:-6              

void retrivevoiceQuery(String Query ,int QueryId )   (Object=Client) 

Method:-7 

 void sendTextQuerytoserver(Query Id,Query)           (Object=server) 

 Method:-8 

 void sendvoiceQuerytoserver(Query Id,File F)               (Object=server) 

 Method:-9        

 void saveinFaqserver(queryId,Query,File F)               (Object=server) 

Method:-10           

 void getQueryFordelete(queryId,username)                 (Object=server) 

Method:-11 

 void deleteQuery(queryId,username)                        (Object=server) 
 
/*User can delete his own Query Which is in Queue */          

(a)boolean  checkqueryStatus(queryId)        (Object=server) 

    /* check  query is retreive or not  */ 

            (b)  If  (a) is true then (d) else (c) 

            (c) void removeQuery(queryId)                        (Object=server) 

            (d) void displaymessage(Text5)                       (Object=GUI)
\subsection*{\underline{ftp utils}}

File Transferring from Instructor to all of Students 

Method:-1 

	void selectFileForTransfer(String files)                   (Object=utils) 

Method:-2  

	void saveFileForTransfer(String files, String path)            (Object=utils) 
Method:-3 

void sendFileToMulticastServer(String path, Vector files, String multicast key, int port)(Object=utils) 

Method:-4 

	void getFilesFromMulticastServer(String multicastkey, int port)   (Object=utils) 

Method:-5 


	void saveFileFromReceiver()  (Object=utils)
\begin{center}
\input{brihSync_ftp.latex}
\label{figure:brihSync_ftp.latex}
\end{center}
\subsubsection*{\underline{Getting Already Sent Files From Server}}


Methods:-6    

void sendRequestToServer(int action)   (Object=server) 

Methods:-7    

perform  Methods:-3 

Methods:-8    

perform   Methods:-4 

Methods:-9 

 perform   Methods:-5 

\subsection*{\underline{Logout}}

    /* When User Press the Logout Button*/ 

Method:-1 

void getLogout()                            (Object=server) 

Method:-2 

void logout(String username,String key,String host)    (Object=SessionManager) 

Method:-3 

void removeAuthorizedUser(String username,String Ip)   (Object=SessionManager) 

Method:-4 

void setLoggerForAuthUser(String username,String jp)   (Object=Logger) 
 
Get a Login Window */ 
\begin{center}
\input{brihSync_Logout.latex}
\label{figure:brihSync_Logout.latex}
\end{center}
\subsection*{\underline{GUI}}

Red coloured comment are the ones which modifies the document. 

Blue coloured portions are the addtions.
 **** End of the Document ****
\begin{center}
\input{brihSync_GUI.latex}
\label{figure:brihSync_GUI.latex}
\end{center}

Written by :- Rekha Pal
\end{document} 

